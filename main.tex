\documentclass[preprint]{aastex}
%\usepackage{draftwatermark}
%\usepackage[printwatermark]{xwatermark}
\usepackage{xcolor,tikz}

\usepackage{hyperref}
\usepackage{breqn}
\usepackage{listings}
\usepackage{color}
\usepackage[utf8]{inputenc}


%\SetWatermarkFontSize{5cm}
%\SetWatermarkColor{red}
%\SetWatermarkText{DRAFT \today}
%\newwatermark*[allpages,color=red!50,angle=45,scale=3,xpos=0,ypos=0]{{\Huge DRAFT}\\ {\normalsize \today}}



% \newsavebox\draftbox
% \savebox\draftbox{\tikz[color=red,opacity=0.25]\node{DRAFT};}
% 
% \newsavebox\datebox
% \savebox\datebox{\tikz[color=red,opacity=0.25]\node{\today};}
% 
% 
% \newwatermark*[
%   allpages,
%   angle=52.31,
%   scale=6,
%   xpos=-35,
%   ypos=40
% ]{\usebox\draftbox\\[-30pt] \usebox\datebox}
% 

\definecolor{codegreen}{rgb}{0,0.6,0}
\definecolor{codegray}{rgb}{0.5,0.5,0.5}
\definecolor{codepurple}{rgb}{0.58,0,0.82}
\definecolor{backcolour}{rgb}{0.98,0.98,0.98}
 
\lstdefinestyle{mystyle}{
    backgroundcolor=\color{backcolour},   
    commentstyle=\color{codegreen},
    keywordstyle=\color{magenta},
    numberstyle=\tiny\color{codegray},
    stringstyle=\color{codepurple},
    basicstyle=\ttfamily\footnotesize,
    breakatwhitespace=false,         
    breaklines=true,                 
    captionpos=b,                    
    keepspaces=true,                 
    numbers=left,                    
    numbersep=5pt,                  
    showspaces=false,                
    showstringspaces=false,
    showtabs=false,                  
    tabsize=2
}
 
\lstset{style=mystyle}
\hypersetup{colorlinks=true}

\begin{document}

\title{Current \textit{HST} Slitless Spectroscopy Analysis Software}

% Note: This document is from the WFC3 team, specifically from the people workign on grism in WFC3.
\author{STScI WFC3 Grism Group \\ (Members: G. Brammer, N. Pirzkal, R. E. Ryan)}

\begin{abstract}
    
    We describe three independently-developed software packages for reducing and analyzing slitless spectroscopic data from the Hubble Space Telescope.  The first, \texttt{aXe}, was developed by the ST-ECF for ACS and later adapter for use with NICMOS and WFC3 data,  and is officially supported by STScI for the user community.  The other two packages were developed for personal use by N. Pirzkal and by G. Brammer.  In this document we describe and compare the capabilities, key assumptions, outputs and runtime of the three separate software packages, which define the current state-of-the art for the analysis of space-based slitless spectroscopic data.
    
% RER: I notice none of the examples discuss contamination or show overlapping dispersions.  
%      Why is this?  Not to air the dirty laundry?  Maybe could we add one section at the very 
%      end with a title like: "Improvements for the Next Generation" or something. And, there
%      show figures for overlaps and other problems?  (Ok, Fig 3 shows a very faint overlap, but
%      I would wager most casual readers won't notice this.)

    
    
    
\end{abstract}

\textit{Draft version: \today}

% N. Pirzkal, R. E. Ryan
\section{Grism data}



% RER: I would write it like this...

Grism observations are obtained by inserting a dispersive element in the light path.  In normal imaging mode, the detector provides a pixelized sampling of some scene:
\begin{dmath}
I(x,y) = \int f(x,y,\lambda)\,\mathcal{S}(\lambda)\,d\lambda
\end{dmath}
where $I(x,y)$ refers to the direct image, $f(x,y,\lambda)$ is intrinsic spectrum at each pixel in the scene, and $\mathcal{S}(\lambda)$ is the sensitivity curve (which in principle depends on detector position as well). The dispersive element transforms $f(x,y,\lambda)$ by redistributing the $\lambda$ dimension.  As described below, this transformation is field-dependent and has been encoded in a series of spatially-varying polynomials that have been calibrated for both WFC3 and ACS.

In the slitless case, the disperser defines a spectral trace, essentially mapping a given wavelength $\lambda$\ onto a unique $x'$,$y'$\ position on the detector. Such a trace is shown in Figure \ref{fig:1}. Assuming no change in sensitivity as a function of position, and ignoring pixel flat-fielding, a dispersed image can be thought of a the convolution of the direct image with the spectra trace:
\begin{dmath}
G(x,y) =  f(x,y,\lambda) \otimes tr(\lambda)
\end{dmath}

\begin{figure}[!ht]
\centering
\includegraphics[width=7.in]{"Figures/Grism_Equation"}
\caption{A simplified view of the relation between the information provided by the direct image (left), understanding of the dispersive element (middle) and the result of convolving the two. If one ignores flat-fielding and wavelength dependence of the shape of the object, this can be thought of (and simulated using) a 2D convolution.}
\label{fig:1}
\end{figure}

Simulating slitless observations can therefore be simulated starting from a set of direct images, providing information about the field and the objects in that field, and sufficient understanding of the properties of the dispersing element.
All of the current code we describe below currently rely on a set of direct observations of the field, or at least a catalog of objects, complete with their size and brightness. 

\section{Use of Simulations}
Currently, we are aware of simulations of grism observations to have been used for the following purpose:

1) To obtain accurate count rate estimates for faint sources and hence serve as a realistic ETC
The WFC3 Exposure Time Calculator supports the G102 and G141 grism modes but does so in an approximate manner. Users have found in the past that to account for the signal dillusion that occurs with different object morphologies, a full 2D simulation is preferred. 

2) To estimate the amount of contamination in extracted spectra
Contamination is a important problem to solve with slitless spectroscopy. The lack of slits results in every source in the field producing spectra (more than one since multiple orders are present). Packages such as \texttt{aXe} uses simulations to estimate this contamination. The quality of this estimate is ultimately limited by the amount of information that is available for every single source in the field (e.g. their location, morphology and spectral characteristics). Simulations have hence been an intrinsic part of the extraction and calibration process from the beginning, although the process has been in large part hidden from users.

3) To examine contamination for a given position angle on the sky 
The amount of contamination affecting  the spectrum of a source is determined by the brightness, location and spectra characteristics of neighboring sources

4) To directly fit models to 2D spectra to estimate redshift and other galaxy stellar population models





\section{Code description}

There are currently three different simulation codes being used within the WFC3 Grism Group:  \texttt{aXe}/ \texttt{aXeSIM}, \texttt{threedhst}, and \texttt{NPSpec}. All three use the same WFC3 calibration products that describe the dispersion and sensitivity of the WFC3 grisms. \texttt{aXe}/\texttt{aXeSIM} can handle grism and prism (e.g. highly non-linear wavelength dispersion). \texttt{aXe}/\texttt{aXeSIM} are both available to the users. \texttt{threedhst}, which is Python based, was developed by the 3D-HST team and is publicly available but not supported at STScI. \texttt{NPSpec} is a Python implementation of simulation kernel that allows one to perform quick simulations for the purpose of SED fitting. The latter is not directly available to the user and was used to gain experience with direct 2D SED fitting of observed spectra.

The three codes described described in this document share common configuration files (e.g. \texttt{aXe} Configuration Files) that define how a given pixel in the direct image is dispersed by the various grism orders (the ``trace''), both geometrically (the location of the trace on the image) and spectro-photometrically (the intensity of the spectrum in electrons per second per pixel). 

The geometric parameters that define the trace relative to a reference pixel in the direct image are 
\begin{itemize}
    \item \texttt{DYDX}: Pixel offset in $y$ of the center of the trace as a function of the $x$ distance from the reference pixel (see \S\ref{eq:4}).
    
    \item \texttt{DLDP}: Effective wavelength of the pixel defined \textit{along} the trace (see \S\ref{eq:6}). 
\end{itemize}

These parameters are stored in the configuration files as polynomials whose coefficients can vary in both $x$ and $y$ dimensions across the instrumental field of view.  \textit{The polynomials are defined in the nominal distorted frame of the \textit{HST} science instruments} (i.e., FLT images produced by the ACS and WFC3 calibration pipelines). A more complete description of these polynomials is presented in Appendix \ref{sec:axeconf}.

The flux calibration of the spectral orders is defined in separate sensitivity files with the sensitivity and its corresponding uncertainty in units of $f_\lambda$ ($\mathrm{erg/s/cm^2/\AA}$) provided as a function of wavelength.  Though the grism dispersion can vary across the instrumental field of view, the sensitivity curve is currently assumed to be constant across the ACS fields. In WFC3, the variation of the sensitivity across the field of view is corrected by the flat-fielding.



% N. Pirzkal
\subsection{The two approaches of current software}
The simulation software currently available approaches the problem in two fundamentally different ways: Either dispersed the entire object at once (Object Based), or disperse each input pixel separately (Pixel Based).

\subsubsection{Object based}

The Object-Based method generates a simulated dispersed spectrum of a source by convolving the trace and the footprint of the object. In this approach, the one dimensional spectrum of the source is first generated in units of $e^-/s$. This spectrum is computed by multiplying the assumed intrinsic spectrum of the object by the sensitivity function (XXX notation..). This spectrum is dropped onto an array, using the DYDX relation and then simply convolved with the footprint of the object. This footprint can take several form. It can be a simple Gaussian description of the objects (based for example on the SExtractor A\_IMAGE, B\_IMAGE, THETA\_IMAGE parameters). It can also be a normalized stamp image of the object (generated by combing a SExtractor segmenation map and a direct image of the object). The convolution can then be either direct, which allows for changing the convolution kernel (i.e. object shape) as a function of wavelength, or can be FFT based.
Figure \ref{sim:1} illustrates this method. Currently, \texttt{aXe}/\texttt{aXeSIM} and \texttt{NPspec} use this method. 


\begin{figure}[!t]
\centering
\includegraphics[width=7.5in]{"Figures/object_sim"}
\caption{This example shows the process (Panels 1 through 4) of Object Based simulation, whereby the footprint of the object, shown on the left, is convolved with the spectral trace.}
\label{sim:1}
\end{figure}

\subsubsection{Pixel based}

The other approach, Pixel Based, consists of each pixel individually. Each pixel is treated as a separate entity and the trace and wavelength relation is obtained and mapped onto the simulated image. This allows for individual pixels to have different colors/spectra as a different spectrum can be assigned to each object. 

\begin{figure}[!ht]
\centering
\includegraphics[width=7.5in]{"Figures/pixel_sim"}
\caption{This example shows the process (Panels 1 through 4) of Pixel Based simulation, whereby individual pixels, shown on the left, are convolved with the spectral trace.}
\label{sim:2}
\end{figure}

% N. Pirzkal
\subsection{``\texttt{aXe}/\texttt{aXeSIM}/\texttt{NPSpec}''} \label{sec:npspec}
The extraction package \texttt{aXe} contains all the code required to generate accurate simulations of a field. \texttt{aXe}, by design, is a end-to-end solution to spectral extraction and calibration. 
\texttt{aXe} was developed by N. Pirzkal at ST-ECF (Space Telescope European Coordinating Facility) in 2001 specifically to extract and calibrate ACS sitless spectra (WFC, HRC, grism and prims). It was further improved at ST-ECF by M. K\"{u}mmel until it was formally handed to STScI/OED  when ST-ECF was shut down. aXe was designed to be a generalized, pipeline-able packaged written in C that followed on the footsteps of \texttt{NICMOSlook} and \texttt{CALNIC-C}, developed by W. Freudling. Many of the features found in \texttt{aXe} were based on lessons learned from \texttt{NICMOSlook}.
As such, its input are high level products, such as Astrodrizzled mosaics of the field and \texttt{SExtractor} catalogs. The results it produces are equally high level and are 2D and 1D extracted and calibrated spectra. The \texttt{aXe} package is meant to be run as a black-box and the user has little to no control on the low level operations \texttt{aXe} performs and there are only a limited number of ways the \texttt{aXe} operations can be customized with our without the addition of custom code. The \texttt{aXe} process is expected to be essentially hands-off and to take a significant amount of time (ten of minutes to hours).  \texttt{aXeSIM}, based on \texttt{aXe}'s internals, is based on the same principle. \texttt{aXeSIM}, using its Objects Based approach, offers a comprehensive approach to generate dispersed simulations of an entire field, with thousands of objects if necessary. Its running time is therefore long. \texttt{aXeSIM} was not designed to generate single object simulations quickly and is therefore less than optimal to do this.
The Objects Based approach was also implemented in pure Python code, \texttt{NPSpec}. This implementation was created with the specific task of 2D in-situ fitting of stellar model spectra (e.g. Bruzual and Charlot models) to an observed spectrum in an \textit{HST} pipeline calibrated FLT file. The main goal was to create a small kernel of Python code that could create reasonable 2D simulation of a spectrum that could then be compared to the observation and used as part as a larger Markov Chain Monte Carlo approach. The code followed the same approach as \texttt{aXe} and produced similar results, using either stamp images and direct convolution or gaussian kernels and FFT convolution. This prototype showed that a pure Python approach could easily generate a simulated spectrum in approximately 30 ms, when using FFT and gaussian kernels.


\begin{itemize}
\item aXeSIM example
\item Description of needed inputs
\item Sample code?
\end{itemize}

%G. Brammer
\subsection{``\texttt{threedhst}''}

G. Brammer developed a pipeline independent of \texttt{aXe} (but relies on the \texttt{aXe} configuration files)  to process the WFC3/IR G141 spectra obtained by the 3D-HST treasury program (GO 12177 \& 12328; PI: van Dokkum; see Brammer et~al. 2012).  At the heart of this pipeline, dubbed \texttt{threedhst}, is the ``Pixel-Based'' approach demonstrated in Fig.~\ref{sim:2} with the spatial template taken from either the direct imaging itself or, optionally, an aligned reference image (for example, a deeper wide-field mosaic).  In practice, the pipeline presently assumes pixels grouped within a segmentation region of a particular object share a common spectrum, so in some sense it is a hybrid of the two methods described above.  However, it should be straightforward (i.e., additional bookkeeping) to extend the algorithm for an arbitrary separation of independent spatial components (e.g., a bulge and a disk of a distant galaxy).

% RER: Note on the previous paragraph.  This is how I'm framing the problem too.  But, I think you 
%      might add just a simple line of butt-cover regarding this decision.  At first blush, this 
%      seems like a restrictive assumption, but I don't think that's true (you experts might disagree).
%      It would be trivial to carve up a segmentation region with some logic.  Maybe the logic is 
%      astrophysical and you carve based on morphology (such as bulge, arms, etc.) or it's 
%      "computational" and you carve based on pixels (like say a 3x3 regridding or Voronoi tesselations).
%      I know the code likely doesn't do this already, but it would seem to be a small preprocessing 
%      module that cuts up a segmentation map and adds appropriate entries in a catalog.



The \texttt{threedhst} pipeline has been designed for speed and flexibility in generating model two-dimensional grism spectra generated from arbitrary input spectra, for example, a galaxy continuum plus emission line template.  The model spectra can then be compared directly to the observed 2D grism spectra, and fits can be evaluated based on the individual pixel uncertainties as defined by the instrument noise model and unaffected by the correlated noise that results from drizzling.  The computation of a model 2D WFC3/G141 spectrum from a higher-resolution input template takes $\sim$3.5 ms for a (NY, NX) = (16, 143) spectral cutout (2$^{\prime\prime}$ along the $y$ spatial axis).  The pipeline includes an automated redshift fitting algorithm that can account for additional redshift constraints from broad-band photometry of a given object.  The algorithm, which evaluates galaxy continuum plus emission-line template fits computed on a high-resolution redshift grid takes $\sim$10 s per object, with the runtime depending on the number of spectral templates and the size of the thumbnail used as the spatial kernel.  We have recently explored implementing the Object-Based model-generation method in the \texttt{threedhst} pipeline, finding substantial speed improvement in generating the model spectra:  models for the (16,143) spectral cutout can be generated in just 0.2 ms.  These models are essentially indistinguishable from those computed with the Pixel-Based approach, given that both methods assume that all pixels within the spatial kernel share the same spectrum.  

\section{Example comparison}

We have run the three grism analysis pipelines on a common dataset to facilitate comparison between them.  The comparison dataset, IBHM52\footnote{\url{http://archive.stsci.edu/hst/search.php?action=Search&sci_data_set_name=IBHM52*}}, is taken from the 3D-HST program (GO 13238).  This visit contains four dithered direct image exposures in the F140W filter (203 s) each followed by a single G141 exposure (1103 s) taken at the same position without moving the telescope.  The pointing lies within the larger COSMOS field imaged in a variety of ACS and WFC3 bands.  Using the larger image mosaics of the COSMOS, all three codes can be used to model the grism spectra of objects near the edges of the field of view whose direct images fall off of the F140W imaging of the visit itself.
    
    Examples of two-dimensional spectra of stars and galaxies as modeled with the different codes are shown in Figs.\ref{fig:star2D} and \ref{fig:gal2D} below.  The top two panels show the observed direct and grism images.  Note that the direct image has been shifted for the display; the position of the direct image in the FLT frame is $\sim$40 pixels to the left of the start of the G141 1st order spectrum.  The subsequent panels show residuals computed with the \texttt{aXeSIM} and \texttt{threedhst} pipelines.  With \texttt{aXeSIM}, we compute models for both Gaussian spatial profiles (``gauss'') and for spatial profiles as extracted from the direct image itself within a segmentation region around an object (``seg'').  The spectral shape is also modeled in two ways with \texttt{aXeSIM}, the first assuming a flat $f_\lambda$ spectrum with the normalization determined from the F140W direct image, and the second (``All'') with a continuum shape determined from the additional ancillary imaging available in this field (WFC3/IR F125W, F140W, and F160W).  The \texttt{threedhst} model uses the segmentation-based approach and refines the spectral shape based on the observed spectrum itself.  \textit{The segmentation-based approach is clearly preferable to assuming Gaussian object profiles}, and even more so for point sources.  
    
    
    
\begin{figure}
    \epsscale{1.0}
    \plotone{Figures/compare_model_star.pdf}
    \caption{2D spectrum of a star.  The top panel shows the F140W image offset to put the object at the center of the panel.  The second panel shows the observed G141 spectrum.  The red outline shows the subset of pixels used in the distribution of residuals shown in the lower right panel.  The bottom five 2D spectrum panels show the $observed-model$ residuals for the modeling codes and input assumptions, as labeled.  The top right panel shows the spatial profile of the 2D spectrum collapsed along the $x$ (wavelength) axis for the observed (filled gray) and model (colored lines) spectra.}
    \label{fig:star2D}
\end{figure}

\begin{figure}
    \epsscale{1.0}
    \plotone{Figures/compare_model_galaxy3.pdf}
    \caption{2D spectrum of an extended galaxy.  The panels are the same as Fig.\ref{fig:gal2D}.}
    \label{fig:gal2D}
\end{figure}

% RER: I'm a bit puzzled by these two figures.  Why does the extended source produce better 
%      residuals than the point source?  Maybe the residuals are actually better for the star, 
%      and it's just graphics that make it look the other way.  I guess I would've thought the star 
%      would be the easiest thing to get right --- high S/N, TRULY a single spectrum per source, 
%      and well-known spatial profile.  But maybe the steepness of the profile *IS* the problem? 
%      Where if you're off just 1% it's like this: (S - 0.01*S)/N = still large ?



\section{Thoughts and Plans for the Next Generation Software}\label{sec:nextgen} 
% this paragraph was written by RER on Oct 13, 2014.

A key limitation to the methods presented above is dealing with deblending (or contamination) of overlapping dispersions and optimally combining information from multiple orientations or positions.  
While \texttt{aXe} uses the same engine as \texttt{aXeSIM} to quantitatively estimates the amount of spectral contamination, this information is not used to deblend, or clean, calibrated spectra. This information is instead provided solely as a quantitative estimate. At present, the WFC3 Grism Group is exploring two different (but related) methods based on their computational feasibility, fundamental assumptions, and limiting systematics.  The first method ({\it forward modeling}) essentially seeks to predict a given dispersed image based on the properties of the sources (such as number/position of sources and their spectra).  These properties are optimized by comparing with the observed dispersed image, but typically the positions are held fixed (to those seen in the direct image).  The second method ({\it linear reconstruction}) attempts to reconstruct the optimal spectra by inversion using least-squares techniques.  We feel each approach has unique pros and cons, and neither seems to inherently superior at this stage.  We have just begun developing prototypes of these tools which are not necessarily built using previously discussed pipelines or configurations, indeed we leave open the possibility that new calibrations, reference files, or descriptions of the grism images will be needed for these tools.  

% RER: I'm reluctant to give timelines and prefer not to (unless asked to do so).  In all honesty, I 
%      I think I'll be fully simulating images in week or so, inverting the problem in a month or so,  
%      reducing simulated data in 2-3 months, and tackling real data in <~6 months.  But I'd prefer to 
%      keep that a secret until I'm a bit more confident of that timeline.  
%
%      As for a pure forward model, I think I could divert my attention and have something in 1-2 month.
%      I'm not sure what your guys timeline is.
%
%      Yes, I'm working in IDL still.  I assure you that's temporary.  John asked me to get something 
%      sooner than later, so once he knows I've got something working I'll translate it to python
%      (with your help, of course ;) ).  I speak python at a 5th grade level, so I really don't see 
%      it as too bad --- FWIW I've already identified places where IDL is bad for this and python will 
%      be better.




\newpage

\appendix

\section{The aXe Configuration File}\label{sec:axeconf}
\texttt{aXe}, \texttt{aXeSIM}, \texttt{NPspec} and \texttt{threedsht} rely on an \texttt{aXe} configuration file to describe how to simulate and extract \textit{HST} slitless spectra. The aXe configuration file was designed with extraction in mind, as opposed to simulation, and there are some minor disadvantages of its current implementation. It is based on the assumption that every aspect of the instrument calibration can be parametrized as $n^{th}$\ order polynomial. The field dependence (how the dispersion varies across the detector) is also assumed to be smooth varying and to be approximated as a 2D polynomial function of the $m^{th}$\ order. While multiple orders are treated independently in a single aXe configuration file, a separate aXe configuration file is required for each detector of each instrument. The two WFC detectors on the ACS instrument, for example, are treated independently. This is usually not a problem as the dispersion is in the x-direction while the detectors are stacked in the y-direction. aXe does not attempt to combine spectra obtained using both detectors, when large dithers in the y-direction are used for example.



For the purpose of simulations, the \texttt{aXe} configuration file is used to identify the various spectral orders to simulate (which are referred as ``BEAMs'' in the \texttt{aXe} configuration file). For a given order (BEAM ``A'' is usually associated with the brightest $+1^{st}$\ order), there are two parameters of interest: DYDX and DLDP. They describe the spectral trace and its wavelength dependence, respectively. DYDX produces the expected offset $dy$ in the $y$-direction with respect to the $y$-position of the source (as measured without the dispersing element) as a function of $dx$, the offset between source and a given column in the $x$-direction. This is shown in Figure \ref{fig:trace}.

\begin{dmath}
dy = a_0 + a_1 \cdot dx + a_2 \cdot dx^2 + ... + a_n \cdot dx^n \label{eq:1}
\end{dmath}

where $dy$\ is the offset (in pixels) between the centroid of the trace and the $y$ position (pixel~$j$) of the source, $dx$\ is the offset in the $x$-direction between a particular pixel on the trace and the $x$-position (pixel~$i$) of the source (Figure \ref{fig:trace}). The order of the polynomial is $n$\ and a simple linear dispersion is obtained with $n=1$. If the instrument had no field dependence and dispersed all spectra in the same way all over the field of view, the coefficients $a_n$ would be simple constants. In the case of WFC3 however, there is a significant amount of field dependence of the trace and this is parametrized by allowing each value of $a_n$ to itself be a 2D polynomial that is a function of the source position ($i$,$j$) on the detector.  In the case of a second order 2D field dependence, we have
\begin{dmath}
a_n(i,j) = b_{n,0} + b_{n,1} \cdot i + b_{n,2} \cdot j + b_{n,3} \cdot i^2 + b_{n,4} \cdot i \cdot j + b_{n,5} \cdot j^2 \label{eq:2}
\end{dmath}
A second order 2D field dependent representation of a simple linear ($n=1$) dispersion therefore becomes
\begin{dmath}
dy(i,j,dx) = b_{0,0} + b_{0,1} \cdot i + b_{0,2} \cdot j + b_{0,3} \cdot i^2 + b_{0,4} \cdot i \cdot j + b_{0,5} \cdot j^2 + dx~(b_{1,0} + b_{1,1} \cdot i + b_{1,2} \cdot j + b_{1,3} \cdot i^2 + b_{1,4} \cdot i \cdot j + b_{1,5} \cdot j^2) \label{eq:3}
\end{dmath}
where we have explicitly expressed $dy$ as a function of the source position ($i$,$j$) and of the $x$-offset ($dx$) between the source and a particular point on the spectrum.

\begin{figure}[!t]
\centering
\includegraphics[width=6.5in]{"Figures/trace_fig"}
\caption{The slitless trace. The object, as it would be visible using a broad band filter, is shown at the bottom left and has pixel coordinates of ($i$,$j$). We also show on the same Figure the resulting spectrum as produced by the G141 grism. The measurements quantities $dx$ and $dy$ used in Equations \ref{eq:1}, \ref{eq:2} and \ref{eq:3} are shown.}
\label{fig:trace}
\end{figure}

\begin{figure}[!t]
\centering
\includegraphics[width=6.5in]{"Figures/wave_fig"}
\caption{The wavelength calibration. The wavelength along the trace is computed as a function of the dp, the arc along the trace from the reference pixel (point on the trace with the same x-coordinate as the reference pixel) and a given position on the trace.}
\label{fig:wave}
\end{figure}


The description of the wavelength dependence {\em along} the trace DLDP is similar and parametrizes the difference in wavelength between the reference point on the trace and a given position on the trace.

\begin{dmath}\label{eqn:lam}
\lambda = \alpha_0 + \alpha_1 \cdot dp + \alpha_2 \cdot dp^2 + ... + \alpha_n \cdot dp^n \label{eq:4}
\end{dmath}
where again, a 2D field dependence implies
\begin{dmath}
\alpha_n(i,j) = \beta_{n,0} + \beta_{n,1} \cdot i + \beta_{n,2} \cdot j + \beta_{n,3} \cdot i^2 + \beta_{n,4} \cdot i \cdot j + \beta_{n,5} \cdot j^2 \label{eq:5}
\end{dmath}
and we have, in the case of a linear wavelength dispersion solution,
\begin{dmath}
d\lambda(i,j,dp) = \beta_{0,0} + \beta_{0,1} \cdot i + \beta_{0,2} \cdot j + \beta_{0,3} \cdot i^2 + \beta_{0,4} \cdot i \cdot j + \beta_{0,5} \cdot j^2 + dp~(\beta_{1,0} + \beta_{1,1} \cdot i + \beta_{1,2} \cdot j + \beta_{1,3} \cdot i^2 + \beta_{1,4} \cdot i \cdot j + \beta_{1,5} \cdot j^2) \label{eq:6}
\end{dmath}

The aXe configuration file encodes the $b_{n,m}$\ parameters of Equation \ref{eq:3} (describing the trace) in DYDX and the $\beta_{m,m}$\ parameters of Equation \ref{eq:6} (describing the wavelength dispersion) in DLDP.  To be precise, $dp$ in the preceeding equations is the arclength along the trace, given as:

\begin{dmath}
dp(X) = \int_{0}^{X} \sqrt{1+y'(x)^2}\, dx
\label{eq:dldp}
\end{dmath}
where $y'(x) = dy/dx$.  For $n\leq2$, this integral has an analytic solution, which we 
utilize in the example code below.  For higher orders, we evaluate this integral numerically (which 
can be very slow).





\section{Code to compute the spectral trace}\label{sec:examplecode}
%\lstlistoflistings
%\newpage
We provide the Python code below to demonstrate reading the \texttt{aXe}-formatted configuration files and computing the spatially-dependent dispersion parameters DYDX and DLDP (\S\ref{sec:axeconf}).

\vspace{1cm}

\lstinputlisting[language=Python]{grism.py}




\end{document}